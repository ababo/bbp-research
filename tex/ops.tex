\documentclass{article}
\usepackage{amsmath}

\setlength{\parskip}{8pt}

\begin{document}

\section{Row Activation}
The \textbf{Row Activation} operation is defined for two Boolean matrices \( X, W \) of size \( m \times n \). It is denoted as:
\[
Z = A(X, W)
\]
and produces a \( 1 \times m \) Boolean row vector \( Z \), where each element \( Z_{1i} \) (for \( i = 1, 2, \dots, m \)) is computed as:  
\[
Z_{1i} = \bigvee_{j=1}^{n} \left( X_{ij} \wedge W_{ij} \right).
\]
In other words, \( Z_{1i} \) is true if there exists at least one column \( j \) such that both \( X_{ij} \) and \( W_{ij} \) are true.

The \textbf{Row Activation} operation can also be applied when one of the arguments (either \( X \) or \( W \)) is a \( 1 \times n \) Boolean row vector, while the other remains an \( m \times n \) matrix. In this case, the vector is broadcast to an \( m \times n \) matrix by repeating it across all \( m \) rows before applying the standard row activation operation.

\section*{Activation Sensitivity}
The \textbf{Activation Sensitivity} operation is defined for two Boolean matrices \( X, W \) of size \( m \times n \). It is denoted as:
\[
G = S(X, W)
\]
and produces an \( m \times n \) Boolean matrix \( G \), where each element \( G_{ij} \) (for \( i = 1, 2, \dots, m \), \( j = 1, 2, \dots, n \)) is computed as:  
\[
G_{ij} = \left( A(X, W) \neq A(X^{(ij)}, W) \right),
\]
where \( X^{(ij)} \) denotes the matrix \( X \) with the element \( X_{ij} \) flipped.

In other words, \( G_{ij} \) is true if changing \( X_{ij} \) alters the result of the \textbf{Row Activation} operation. This measures the sensitivity of each input element to the row activation output.

Similarly to \textbf{Row Activation}, the \textbf{Activation Sensitivity} operation can also be applied when one of the arguments (either \( X \) or \( W \)) is a \( 1 \times n \) Boolean row vector, while the other remains an \( m \times n \) matrix. In this case, as before, the vector is broadcast to an \( m \times n \) matrix by repeating it across all \( m \) rows before applying the standard row activation operation.

\end{document}
