\documentclass{article}
\usepackage{amsmath}

\setlength{\parskip}{5pt}
\linespread{1.1}

\begin{document}

\section{Row Activation}
The \textbf{Row Activation} operation is defined for a Boolean row vector \( X \) of size \( 1 \times n \) and a Boolean matrix \( W \) of size \( m \times n \). It is denoted as:
\[
Z = \mathcal{A}(X, W)
\]
and produces a \( 1 \times m \) Boolean row vector \( Z \), where each element \( Z_{1i} \) (for \( i = 1, 2, \dots, m \)) is computed as:
\[
Z_{1i} = \bigvee_{j=1}^{n} \left( X_{1j} \wedge W_{ij} \right).
\]
In other words, \( Z_{1i} \) is true if there exists at least one column \( j \) such that both \( X_{1j} \) and \( W_{ij} \) are true.

\section{Activation Sensitivity}
The \textbf{Activation Sensitivity} operation is defined for two Boolean matrices \( A, B \) of size \( m \times n \). It is denoted as:
\[
C = \mathcal{S}(A, B)
\]
and produces an \( m \times n \) Boolean matrix \( C \), where each element \( C_{ij} \) (for \( i = 1, 2, \dots, m \), \( j = 1, 2, \dots, n \)) is computed as:
\[
C_{ij} = \left( \mathcal{A}(A, B) \neq \mathcal{A}(A^{(ij)}, B) \right),
\]
where \( A^{(ij)} \) denotes the matrix \( A \) with the element \( A_{ij} \) flipped.

In other words, \( C_{ij} \) is true if changing \( A_{ij} \) alters the result of the \textbf{Row Activation} operation. This measures the sensitivity of each input element to the row activation output.

The \textbf{Activation Sensitivity} operation can be applied when one of the arguments (either \( A \) or \( B \)) is a \( 1 \times n \) Boolean row vector, while the other remains an \( m \times n \) matrix. In this case, the vector is broadcast to an \( m \times n \) matrix by repeating it across all \( m \) rows before applying the standard row activation operation.

\section{Error Projection}
The \textbf{Error Projection} operation is defined for a Boolean matrix \( S \) of size \( m \times n \) and a Boolean row vector \( E \) of size \( 1 \times m \). It is denoted as:
\[
E' = \mathcal{P}(S, E)
\]
and produces a \( 1 \times n \) Boolean row vector \( E' \), where each element \( E'_{1j} \) (for \( j = 1, 2, \dots, n \)) is computed as:
\[
E'_{1j} = \left( \bigvee_{j: E_{1i} = 1} S_{ij} \right) \wedge \neg \left( \bigvee_{i: E_{1i} = 0} S_{ij} \right).
\]
In other words, \( E'_{1j} \) is true if there exists at least one row \( i \) where \( E_{1i} \) is true and \( S_{ij} \) is also true, while ensuring that no row \( i \) where \( E_{1i} \) is false contributes a true value in \( S_{ij} \).

\end{document}
